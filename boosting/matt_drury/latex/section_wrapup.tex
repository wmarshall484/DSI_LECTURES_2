\section{Final Words About Boosting}

\begin{frame}
Gradient Boosting is the best off-the-shelf learning algorithm available today.\\~\\
It effortlessly produces accurate models.\\~\\
\end{frame}
%
\begin{frame}
Nonetheless, it has drawbacks.
\end{frame}
%
\begin{frame}{Drawbacks of Gradient Boosting}
\begin{itemize}

\only<1>{
  \item Boosting creates very complex models.  It can be difficult to extract intuitive, conceptual, or inferential information from them.
}

\only<2>{
  \item Boosting is difficult to explain (maybe you just learned this through experience).  It can be hard to convince business leader to accept such a black box model.
}

\only<3>{
  \item Boosted models can be difficult to implement in production environments due to their complexity.
}

\only<4>{
  \item The sequential nature of the standard boosting algorithm makes it very difficult to parallelize (compared to, for example, random forest).  Recently, there has been great progress (xgboost).
}
\end{itemize}
\end{frame}
%

\begin{frame}{Software Options For Boosting}

\begin{itemize}
  \item There are many quality open source implementations of gradient boosting algorithms:
  \begin{itemize}
    \item Sklearn, obviously :)
    \item R's \texttt{gbm} is great.  It offers many different loss functions
    and is feature rich.
    \item \texttt{xgboost} is a modern interpretation of gradient boosting, with
    many algorithmic innovations that make it faster and more accurate.  State of the art, with python interface.
    \item \texttt{H2O} offers a distributed gradient booster that can handle
    massive data sets.  Also offers a python interface.
  \end{itemize}
\end{itemize}

\end{frame}
%
\begin{frame}{Q\&A}
  \begin{itemize}
    \item Understand the conceptual foundation of Boosting
    \item Understand the algorithm's hyperparameters, and how to tune them.
    \item Understand some basic strategies for interpreting a booster.
    \item Understand the drawbacks of boosting.
    \item Be aware of the possibility of creating your own loss function.
  \end{itemize}
\end{frame}
